\documentclass[manuscript,screen,review]{acmart}

\AtBeginDocument{%
  \providecommand\BibTeX{{%
    Bib\TeX}}}

% \setcopyright{acmcopyright}
% \copyrightyear{2024}
% \acmYear{2024}
% \acmDOI{10.1145/1122445.1122456}


\usepackage{algorithmicx}
\usepackage{algpseudocode}
\usepackage{subfigure}
\usepackage{textcomp}
\usepackage{booktabs}
\usepackage{graphicx}
\usepackage{subfigure}
\usepackage{amsmath}
\allowdisplaybreaks[4]
\makeatletter  
\newif\if@restonecol  
\makeatother  
\let\algorithm\relax  
\let\endalgorithm\relax
\usepackage[linesnumbered,ruled,vlined]{algorithm2e}
\usepackage{algpseudocode}
\renewcommand{\algorithmicrequire}{\textbf{Input:}}%Use Input in the format of Algorithm
\renewcommand{\algorithmicensure}{\textbf{Output:}}%Use Output in the format of Algorithm
\newtheorem{theorem}{Theorem}
\newtheorem{lemma}{Lemma}
\newtheorem{definition}{Definition}
\newcommand{\etal}{\textit{et al}.}
\newcommand{\ie}{\textit{i}.\textit{e}.}
\newcommand{\eg}{\textit{e}.\textit{g}.}

\begin{document}
	
	\title{Chain Disruption Risk-Oriented Task Migration in Multiplex Networked Industrial Chains}
	
	\author{Kai Di, Pan Li, Tienyu Zuo, Fulin Chen, Yuanshuang Jiang}
	\email{{dikai, lipan, tienyu.zuo, chenfulin, yuanshuangjiang}@seu.edu.cn}
	\affiliation{
		\institution{Southeast University}
		\city{Nanjing}
		\country{China}
	}

	\author{Jiuchuan Jiang}
	\email{jcjiang@nufe.edu.cn}
	\affiliation{
		\institution{Nanjing University of Finance and Economics}
		\city{Nanjing}
		\country{China}
	}
	
	\author{Yichuan Jiang}
	\authornote{Corresponding author.}
	\email{yjiang@seu.edu.cn}
	\affiliation{%
		\institution{Southeast University}
		\city{Nanjing}
		\country{China}
	}
	
	\renewcommand{\shortauthors}{Kai Di et al.}
	
	\begin{abstract}
		In industrial production processes, disruptions can occur within the industrial chain. An example of such a disruption occurred during the COVID-19 pandemic, where many production agents were affected by the risk of disruption and unable to participate in the collaboration. To ensure the continuity of industrial production, it becomes necessary to migrate tasks from agents impacted by disruption risk to other agents. Developing task migration strategies for industrial chains that encounter disruption risks necessitates taking into account the emerging characteristics of multiplex networks within these chains. In such a multiplex networked industrial chain, disruption risk in one layer can trigger cascading effects on other layers, resulting in a ripple effect across the entire system. Consequently, task migration in multiplex networked industrial chains presents the following challenges: (1) Disruption risk not only causes mismatches between the capabilities of product agents and tasks but also leads to mismatches between network layers and tasks, thus increasing the dimension of problem space; (2) Disruption risk often occurs simultaneously across multiple product agents and network layers, leading to an increased number of tasks requiring migration, thus amplifying the complexity of the solution space. To address the mentioned challenges, we introduce a novel concept called the “multiplex potential field”, which effectively captures the interdependencies and complexities within multiplex networked industrial chains. Leveraging this concept, we propose a hierarchical contextual task migration algorithm that utilizes the multiplex potential field as a guiding principle for both inter-layer and intra-layer task migrations. Comprehensive experiments are conducted to evaluate our proposed approach, and the results consistently demonstrate that our proposed approach nearly always outperforms over previous benchmark algorithms in terms of achieving higher utility. Moreover, our approach significantly improves the task completion ratio while reducing task execution costs. Notably, our proposed algorithm achieves solutions that are very close to those obtained by the optimal solver CPLEX while requiring significantly less computational time.
	\end{abstract}
	
	\begin{CCSXML}
		<ccs2012>
		<concept>
		<concept_id>10010147.10010178.10010199.10010202</concept_id>
		<concept_desc>Computing methodologies~Multi-agent planning</concept_desc>
		<concept_significance>500</concept_significance>
		</concept>
		</ccs2012>
	\end{CCSXML}
	
	\ccsdesc[500]{Computing methodologies~Multi-agent planning}
	
	\keywords{multiagent collaboration, multiplex network, task migration, industrial chains}
	
	\maketitle
	
	\section{Introduction}
	
	Industrial chains are complex networks of interconnected product agents involved in the production, distribution, and delivery of goods and services \cite{13rding2020transmission,12ryu2021secure}. However, disruptions in industrial chains can occur due to various factors such as natural disasters, geopolitical events, and pandemics \cite{9928221,lv2022digital}. For instance, the COVID-19 pandemic highlighted the vulnerability of industrial chains to unexpected disruptions, leading to supply shortages, production delays, and operational challenges. Such disruptions can severely impact the efficiency, reliability, and stability of the entire chain \cite{9732466,karamoozian2022hybrid}.
	
	In recent years, driven by technological advancements and globalization, industrial chains have evolved into multiplex networked systems. These chains consist of interconnected layers, each with its own agents, resources, and functions, collaborating to fulfill complex production processes \cite{ref2,30rjiang2015reliable}. However, disruptions in one layer may trigger cascading effects on other layers, leading to a ripple effect throughout the entire system. A single disruption can affect multiple network layers simultaneously, leading to an increased number of tasks that need migration and resource reallocation to maintain operational balance and continuity. Therefore, it is crucial to focus on task migration strategies in multiplex industrial chains to reduce the impact of disruptions and ensure efficient operations \cite{ref3,ref4,suen1992efficient}.
	
	The chain disruption risk in multiplex networked industrial chains presents several challenges for designing task migration approaches, which are shown as follows:
	\begin{itemize}
		\item \textit{Dimension Curse.} Disruption risk not only leads to mismatches between the capabilities of product agents and tasks but also results in mismatches between network layers and tasks. As a consequence, this increases the dimension of the problem space, making it more intricate to find optimal solutions.
		\item \textit{Complexity Curse.} Disruption risk often occurs simultaneously across multiple product agents and network layers, causing an increased number of tasks that need migration. Consequently, this amplifies the complexity of the solution space, making it more challenging to devise efficient migration strategies.
	\end{itemize}
	
	Facing the aforementioned challenges, this paper introduces a novel concept called the “multiplex potential field” and builds upon it to develop a hierarchical contextual task migration algorithm. The multiplex potential field captures the interdependencies and complexities present in multiplex networked industrial chains, enabling a more comprehensive representation of the network layers and their interactions. Subsequently, the hierarchical contextual task migration algorithm maximizes the benefits of both centralized control and distributed control. It adopts a hierarchical approach to perform centralized task migration among network layers, aiming to achieve load balancing across the layers. Simultaneously, within each agent and its context, the algorithm implements contextual task migration based on the gradient descent direction within the network layer.
	
	Finally, comprehensive experiments are conducted to compare our presented approach with previous benchmark multiagent task migration approaches \cite{chowdhury2018context,jiang2022batch}. The results demonstrate that our approach consistently outperforms these benchmarks, achieving higher utility and improving the task completion ratio while reducing task execution costs. Notably, our proposed approach yields optimal results that are statistically comparable to those obtained by the optimization solver CPLEX \cite{qin2023salp,guo2022multiobjective,wu2022novel}, but with significantly reduced running time requirements.
	
	To the best of our knowledge, this paper is the \textit{first work} that investigates task migration strategy optimization in multiplex networked industrial chains facing chain disruption risk. The main contributions of this paper are highlighted as follows:
	
	\begin{enumerate}
		\item A novel model is introduced to formulate task migration problems arising from chain disruption risk in multiplex networked industrial chains. This model extends the multiagent task migration problem models from previous studies.
		\item An innovative concept, the multiplex potential field, is introduced as a guiding principle for performing both inter-layer and intra-layer task migrations in multiplex networked industrial chains. It effectively captures the interdependencies and complexities within these systems, offering a comprehensive representation of the network layers and their interactions.
		\item A hierarchical contextual task migration algorithm is developed, leveraging the multiplex potential field as a guiding principle. This algorithm combines the advantages of centralized and distributed control, enabling hierarchical inter-layer task migration with global information and contextual intra-layer task migration in a flexible manner.
		\item Comprehensive experiments are conducted by comparing the proposed algorithm with existing baseline algorithms. The results demonstrate that the algorithm proposed in this paper consistently achieves higher utility in this problem scenario. Moreover, it enhances task completion rates while reducing task execution costs. Notably, the proposed algorithm achieves results comparable to those obtained by optimal solvers, but with significantly reduced time consumption.
	\end{enumerate}
	
	\section{Related Work}
	
	\label{relatedwork}
	In this section, we provide an overview of the relevant literature concerning task migration in simplex networks. Additionally, we discuss the challenges that arise when applying these algorithms to multiplex networked industrial chains.
	
	\subsection{Simplex Networked Task Migration}
	Research on task migration in simplex networks aims to optimize task scheduling, load balancing, and resource allocation \cite{14rzhou2019risk,15rbhatti2016unified}. These studies focus on developing algorithms and techniques to efficiently distribute tasks among available resources, minimize completion time, and enhance resource utilization. Task migration strategies have played a crucial role in achieving these objectives by dynamically reallocating tasks between resources to maintain a balanced load in simplex networks \cite{16rlin2022intelligent,17rchipara2012real}. 
	
	Several researchers have proposed various algorithms and techniques to address task migration in simplex networks. For example, dynamic task assignment strategies have been employed to efficiently migrate tasks among agents and minimize completion time \cite{li2022distributed}. Load balancing techniques, such as task migration between overloaded and underutilized resources, aim to achieve a more balanced distribution of tasks and optimize resource utilization \cite{31rjiang2008contextual,kashani2022load}. Additionally, resource management algorithms optimize resource allocation based on availability and capabilities \cite{ref46}. However, these approaches are primarily designed for simplex networks and may not directly apply to multiplex networks. The unique challenges and interdependencies among network layers in the studied multiplex networked industrial chains call for the development of new approaches and adaptations to existing algorithms.
	
	In conclusion, task migration in simplex networks has been extensively studied to optimize task scheduling, load balancing, and resource allocation. However, the applicability of these algorithms in multiplex networked industrial chains systems needs further investigation. 
	
	
	\subsection{Multiplex Networked Multiagent Cooperation}
	
	% This research focuses on studying how agents in a multiplex networked system can effectively collaborate and coordinate their actions to achieve collective goals. This area of research explores the challenges and opportunities that arise when multiple network layers interact and influence the decision-making and coordination processes of the agents. 
	
	Several researchers have investigated different approaches to address the challenges of multiplex networked multiagent cooperation \cite{32rli2017maintenance,33rli2021task}. One approach is to develop coordination mechanisms that enable agents to share information, exchange messages, and synchronize their actions across network layers. This requires designing communication protocols and information sharing strategies that take into account the heterogeneity and connectivity patterns of the multiplex network \cite{31rjiang2008contextual,32rli2017maintenance}. Another approach is to develop decision-making algorithms that allow agents to adaptively allocate their resources and tasks based on the varying conditions and dynamics of the multiplex network. This involves developing optimization algorithms that consider the trade-offs between local and global objectives, taking into account the constraints and interdependencies across network layers \cite{33rli2021task}. 
	
	Compared to simplex networks, achieving effective cooperation in multiplex networked systems poses challenges due to the 
	increased complexity and interdependencies among the network layers  \cite{31rjiang2008contextual,33rli2021task}. Agents must consider interactions and dependencies across multiple layers when making decisions and coordinating actions. Heterogeneity in node capacities, connection patterns, and resource availability adds another layer of complexity. Moreover, the dynamic nature of the multiplex network, with disruptions, failures, and topology changes, further complicates agent coordination and cooperation \cite{9928221,9732466}.
	
	In summary, this research significantly contributes to advancing multiagent systems and sheds light on the design of intelligent and adaptable systems for efficient cooperation in multiplex networked environments.
	
	\subsection{Challenges in Multiplex Networked Industrial Chains}
	
	\begin{figure*}[htb]
		\centering
		\subfigure[Initial state]{
			\begin{minipage}[t]{0.31\linewidth} 
				\centering
				\includegraphics[width=1.8in]{Figure/fig1a.png}
				\label{simplecasea}
			\end{minipage}
		}
		\subfigure[Migrating all tasks in upper layer]{
			\begin{minipage}[t]{0.31\linewidth}
				\centering
				\includegraphics[width=1.8in]{Figure/fig1b.png}
				\label{simplecaseb}
			\end{minipage}
		}
		\subfigure[Migrating all tasks in lower layer]{
			\begin{minipage}[t]{0.31\linewidth}
				\centering
				\includegraphics[width=1.8in]{Figure/fig1c.png}
				\label{simplecasec}
			\end{minipage}
		}
		
		\subfigure[Treating all layers as one]{
			\begin{minipage}[t]{0.31\linewidth} 
				\centering
				\includegraphics[width=1.8in]{Figure/fig1d.png}
				\label{simplecased}
			\end{minipage}
		}
		\subfigure[Optimal solution]{
			\begin{minipage}[t]{0.31\linewidth} 
				\centering
				\includegraphics[width=1.8in]{Figure/fig1e.png}
				\label{simplecasee}
			\end{minipage}
		}
		\caption{Examples of different types of task migration strategies. The agent marked with red circles is affected by disruption risk.}
		\label{fig1}
	\end{figure*}
	
	To illustrate the limitations of previous task migration methods in multiplex networked industrial chains, we present an example shown in Fig. \ref{fig1}. Previous task migration algorithms fall into two categories, and their specific computation processes are as follows:
	
	\begin{itemize}
		\item \textit{Migrating All Tasks in One Layer:} As shown in Fig. \ref{simplecaseb}, in the upper network layer, the agent affected by disruption risk migrates tasks to an agent unaffected by disruption risk to achieve minimax load balancing within the upper network layer \cite{38rcai2018human,39rwang2019routing,40rhong2023logistics}. Similarly, in the lower network layer, as depicted in Fig. \ref{simplecasec}, common agents migrate tasks to unaffected agents in the lower layer to achieve minimax load balancing within the lower network layer. However, the drawback of this approach is that it does not utilize all the agents in the multiplex network, limiting its adaptability. Therefore, both of these computed results fail to achieve the optimal solution depicted in Fig. \ref{simplecasee} which not only achieves minimax load balancing within both the upper and lower network layers but also minimizes the average load across the layers.
		\item \textit{Treating All Layers as One Layer:} In this approach, as shown in Fig. \ref{simplecased}, task migration occurs across the entire multiplex network to achieve minimax load balancing within the network layer, comprising both upper and lower network layers \cite{34rzuo2019resilient,35rzuo2020resilient,36rji2022fully}. Utilizing all agents in the network, this algorithm considers all layers as a unified entity. However, it overlooks load balancing among network layers, preventing it from reaching the optimal solution obtained in Fig. \ref{simplecasee}. The optimal solution not only achieves minimax load balancing within the upper and lower network layers but also minimizes the average load across all layers.
	\end{itemize}
	
	Overall, the complexity and interdependencies among network layers present unique challenges that require the development of novel approaches and adaptations to existing algorithms.
	
	
	\section{Problem Formulation and Analyses}
	\label{Problem}
	\subsection{Preliminaries}
	
	\textbf{Multiplex Networked Industrial Chain.} In a multiplex networked industrial chain, due to the existence of different types of collaborations, product agents are connected to each other by edges belonging to $M$ different types, where $M$ is used to characterize the number of collaborating types \cite{30rjiang2015reliable,magnani2021community}. The links of the same type and the product agents involved form a network layer. A multiplex networked industrial chain is therefore completely specified by a Laplace matrix $\mathcal{L}=\left[L^{[1]}, L^{[2]},..., L^{[M]}\right]$, the elements of which are the adjacency matrices $L^{[l]} =\{e_{ij}^{[l]}\}$, where $e_{ij}^{[l]} =1$ if product agents $i$ and $j$ are connected by an edge of type $l$ and $+\infty$ otherwise \cite{32rli2017maintenance,liu2018optimizing}.
	
	\textbf{Product Agent.} According to the above definition of multiplex networked industrial chains, the product agents involved can be denoted as a set of agents with $M$ types of links $\mathcal{A}=\left[A^{[1]}, A^{[2]},..., A^{[M]}\right]$, where the set of agents with link type $l$ can be denoted as $A^{[l]}=\{a_1,a_2,..., a_N\}$ \cite{ref44}. In the subsequent paper, we will abbreviate “product agent” as “agent”. Each product agent $a_i\in \mathcal{A}$, it can be characterized by a tuple $<v_i,Q_i,\mathcal{A}_i>$ where:
	\begin{itemize}
		\item $v_i$ represents the execution load capacity of agent $a_i$, indicating that the number of loads executed by $a_i$ in each time unit \cite{ref41};
		\item $Q_i$ represents the task queue associated with agent $a_i$, and $\mathcal{A}_i$ represents the set of network layers to which agent $a_i$ belongs. It is important to note that agent $a_i$ may belong to multiple distinct network layers within the multiplex networked industrial chains \cite{ref43}.
	\end{itemize}
	
	\textbf{Chain Disruption Risk.} In industrial production processes, disruptions in the industrial chain can occur. A motivating example is the COVID-19 pandemic, where many agents were unable to operate under normal conditions \cite{9928221,lv2022digital,9732466,karamoozian2022hybrid}. Disruptions in the multiplex networked industrial chains can impact both the agents and the network links.
	\begin{itemize}
		\item \textit{The impact on agents.} The agents that are exposed to the risk of disruption will be unable to complete their tasks as intended. We can represent the situation of agent $a_i$ using the variable $\gamma_i$, which takes binary values of either 0 or 1. When $\gamma_i=1$, it indicates that agent $a_i$ is affected by the risk of disruption, whereas $\gamma_i=0$ denotes that the agent remains unaffected. In the case where $\gamma_i=1$, agent $a_i$ is unable to execute tasks effectively, with a task size denoted as $v_i=0$ under unit cost. This indicates that the affected agent is incapable of continuing task execution.
		\item \textit{The impact on network links.} The risk of disruption not only affects agents within the multiplex networked industrial chains but also disrupts the connections between these agents. This paper characterizes such disruptions to the links as “link failures” among agents. Next, we present the cost associated with task migration from agents $a_i$ to $a_j$ in the presence of disruption risk,
		\begin{equation}
		c_{ij} = 
		\begin{cases}
		e_{ij}^{\lbrack l\rbrack},& \forall a_{i},a_{j} \in A^{\lbrack l\rbrack},\gamma_j=0\\
		+\infty,& \forall a_{i},a_{j} \in A^{\lbrack l\rbrack},\gamma_j=1\\
		{\underset{m \in \mathcal{A}}{\mathop{\min}}\left\{ c_{im} + c_{mj} \right\},}&{\text{otherwise.}} 
		\end{cases}
		\end{equation}
	\end{itemize}
	
	\textbf{Multi-Level Load.} When a multiplex networked industrial chain is exposed to the risk of disruption, ensuring the smooth operation of the production process requires migrating tasks from affected agents to unaffected agents. In the task migration process, it is crucial to consider not only the load of agents within the multiplex networked industrial chains but also the load conditions specific to their corresponding network layers \cite{ref46,ref45}. We describe the influence of multi-level loads on the task completion rate of agents as follows:
	\begin{equation}
	p_{i} = \overset{\text{agent~level}}{\overbrace{\left( 1 - \psi_{A}\left( S_{i} \right) \right)}} \cdot \underset{\text{network~layer~level}}{\underbrace{\prod\limits_{A^{\lbrack l\rbrack} \in \mathcal{A}_{i}}\left( 1 - \psi_{N}\left( \frac{\sum_{a_{i} \in A^{\lbrack l\rbrack}}S_{i}}{\left| A^{\lbrack l\rbrack} \right|} \right) \right)}}
	\end{equation}
	where $S_i$ represents the load size of agent $a_i$ for task queue $Q_i$. $\mathcal{A}_i$ represents the set of network layers to which agent $a_i$ belongs. $\psi_A(\cdot)$ and $\psi_N(\cdot)$ are both monotonically increasing functions, ranging from 0 to 1, used to describe the respective impacts of the agent level and network level on the task completion rate of agents \cite{ref46,ref45,31rjiang2008contextual}.
	
	\subsection{Problem Formulation}
	\label{iip}
	The objective of this paper is to identify task migration strategies under the risk of chain disruption, aiming to enhance the task completion rate while minimizing the increase in task execution costs. To achieve this objective, the optimization objective function comprises two components: the ratio of task execution cost increase and the task completion ratio. Next, we provide the definitions of these two objective functions.
	\begin{definition}
		Task Execution Cost Increase Ratio \textbf{(TECIR)}:  This objective function measures the ratio of the increase in task execution costs caused by task migration. It is defined as the difference between the total task execution costs after migration $\mathcal{C}(\pi)$ and the total task execution costs before migration $\hat{\mathcal{C}}$, divided by the total task execution costs before migration. The specific mathematical form is as follows:
	\end{definition}
	
	\begin{equation}
	\begin{aligned}
	\mathcal{U}(\pi) & =\frac{\mathcal{C}(\pi)}{\hat{\mathcal{C}}}-1 \\
	& =\sum_{a_i \in \mathcal{A}}\left\{\overbrace{\sum_{t_k \in T} s_k  \cdot  x_i^k / v_i}^{\text {task completion cost }}+\underbrace{\sum_{a_j \in \mathcal{A}}\left(\sum_{t_k \in T} c_{i j} \cdot y_{i j}^k\right)}_{\text {task migration cost }}\right\} / \hat{\mathcal{C}}-1
	\end{aligned}
	\end{equation}
	where $x_i^k=1$ indicates that task $t_k$ is executed by agent $a_i$, and $x_i^k=0$ otherwise. Similarly, $y_{ij}^k=1$ indicates that task $t_k$ migrates from agent $a_i$ to $a_j$, while $y_{ij}^k=0$ indicates no migration. The variable $s_k$ represents the size of task $t_k$, and $v_i$ represents the size of the task executed by agent $a_i$ with unit cost.
	
	
	\begin{definition}
		Task Completion Ratio \textbf{(TCR)}: This objective function quantifies the ratio of successfully completed tasks to the total number of tasks. It is defined as the number of completed tasks after migration divided by the total number of tasks. The specific mathematical form is as follows:
	\end{definition}
	\begin{equation}
	\mathcal{R}(\pi)=1-\left(\sum_{i \in \mathcal{A}_e}\left|Q_i\right|+\sum_{j \in \mathcal{A}_u}\left|Q_j\right| \cdot\left(1-p_j\right)\right) /|T|
	\end{equation}
	where $\mathcal{A}_e=\left\{a_i \mid \forall a_i \in \mathcal{A}, \gamma_i=1\right\}$ denotes the set of agents that are exposed to the risk of disruption, and $\mathcal{A}_u=\left\{a_j \mid \forall a_j \in \mathcal{A}, \gamma_j=0\right\}$ denotes the set of unaffected agents. 
	
	Based on the defined optimization objectives TECIR and TCR, we now present the definition of the \textit{Multiplex Task Migration Problem with Disruption Risk} studied in this paper.
	
	\begin{definition}
		Multiplex Task Migration Problem with Disruption Risk \textbf{(MTMP-DR)}: In the context of a multiplex networked industrial chain with the presence of disruption risk, consisting of a set of agents $\mathcal{A}$ and a multiplex network link structure $\mathcal{L}$, the objective is to design a task migration strategy $\pi$ that aims to enhance the task completion ratio while minimizing the task execution cost increase ratio.
	\end{definition}
	
	Then, the defined MTMP-DR can be formulated as an integer programming model as follows:
	
	\begin{alignat}{2}
	& \max \quad -\alpha\cdot \mathcal{U} + \beta \cdot \mathcal{R} &{}&\tag{MTMP-DR}\label{objective}\\
	\mbox{s.t.}\quad
	&\mathcal{U}=\mathcal{C}/\hat{\mathcal{C}}-1 \label{constraint1}\\
	&\mathcal{C} = {{\sum\limits_{i \in \mathcal{A}}\left( {{\sum\limits_{k \in T}{s_{k} \cdot x_{i}^{k}/v_{i}}} + {\sum\limits_{j \in \mathcal{A}}{\sum\limits_{k \in T}{c_{ij} \cdot y_{ij}^{k}}}}} \right)}} \label{constraint2}\\
	&\mathcal{R}=1-\left(\sum_{i \in \mathcal{A}_e} \sum_{k \in T} x_i^k+\left(1-p_j\right) \cdot \sum_{j \in \mathcal{A}_u} \sum_{k \in T} x_i^k\right) /|T| \label{constraint3}\\
	&p_{i} = \left( 1 - r_{i} \right) \cdot {\prod\limits_{A^{\lbrack l\rbrack} \in \mathcal{A}_{i}}\left( 1 - r^{\lbrack l\rbrack} \right)},\quad\forall i \in \mathcal{A}\label{constraint4}\\
	&r_{i} = \psi_{A}\left( S_{i} \right),\quad\forall i \in \mathcal{A}\label{constraint5}\\
	&r^{\lbrack l\rbrack} = \psi_{N}\left( {\sum_{i \in A^{\lbrack l\rbrack}}{S_{i}/\left| A^{\lbrack l\rbrack} \right|}} \right),\quad\forall A^{\lbrack l\rbrack} \in \mathcal{A}_{i} \label{constraint6}\\   
	&{\sum\limits_{i \in \mathcal{A}}x_{i}^{k}} = 1,\quad\forall k \in T \label{constraint7}\\
	&{\sum\limits_{i \in \mathcal{A}}{\sum\limits_{k \in T}x_{i}^{k}}} \leq |T| \label{constraint8}\\
	&\sum_{k\in T}y_{ij}^k \leq M \cdot (1-\gamma_j), \quad i,j \in \mathcal{A} \label{constraint9}\\
	&x_{j}^{k} = {\sum\limits_{i \in \mathcal{A}}{y_{ij}^{k} \cdot W_{i}^{k}}},\quad\forall j \in \mathcal{A};k \in T  \label{constraint10}\\
	&{\sum\limits_{j \in \mathcal{A}}y_{ij}^{k}} \leq W_{i}^{k},\quad\forall i \in \mathcal{A} \label{constraint11}\\
	&x_{i}^{k} \in \left\{ 0,1 \right\},\quad\forall i \in \mathcal{A};k \in T \label{constraint12}\\
	&y_{ij}^{k} \in \left\{ 0,1 \right\},\quad\forall i,j \in \mathcal{A};k \in T \label{constraint13}
	\end{alignat} 
	
	In the presented integer programming model, the optimization objective is defined by a utility function that integrates two key objectives: TECIR and TCR. The utility function allows for balancing the trade-off between these two objectives by introducing parameters $\alpha$ and $\beta$ (where $\alpha,\beta \geq 0$ and $\alpha+\beta=1$). The constraints involved in the integer programming model can be categorized into the following types:
	
	\begin{itemize}
		\item \textit{Optimization Objectives Constraints:} Constraints \eqref{constraint1}-\eqref{constraint6} specify the primary objectives of the optimization problem and their corresponding constraints.
		\item \textit{Task Assignment Constraints:} Constraints \eqref{constraint7} and \eqref{constraint8} are designed to ensure that each task is assigned to only one agent, thereby preventing any possibility of task duplication in the task migration process.
		\item \textit{Task Migration Constraints:} Constraints \eqref{constraint9}-\eqref{constraint11} regulate the process of task migration between agents. Among these, constraint \eqref{constraint9} prohibits tasks from being migrated to agents that are exposed to the risk of disruption. Constraints \eqref{constraint10} and \eqref{constraint11} describe the assignment of tasks after the task migration strategy has been implemented. Specifically, $W_i^k=1$ indicates that task $t_k$ is assigned to agent $a_i$ prior to the task migration strategy, while 0 indicates otherwise.
		\item \textit{Binary and Integer Constraints:} Constraints \eqref{constraint12} and \eqref{constraint13} ensure that decision variables $x_i^k$ and $y_{ij}^k$ take binary values (0 or 1). Here, $x_i^k=1$ indicates that task $t_k$ is executed by agent $a_i$, and $x_i^k=0$ otherwise. Similarly, $y_{ij}^k=1$ indicates that task $t_k$ migrates from agent $a_i$ to $a_j$, while $y_{ij}^k=0$ indicates no migration.
		
	\end{itemize}
	
	\subsection{Complexity Analysis}
	
	In this subsection, we analyze the complexity of the MTMP-DR problem.
	
	\begin{theorem}
		The MTMP-DR problem is $\mathcal{NP}$-hard.
	\end{theorem}
	\begin{proof}
		The problem is clearly in $\mathcal{NP}$, as one can easily construct a task migration strategy for a given multiplex networked industrial chain and verify its total utility in polynomial time.
		
		To prove its $\mathcal{NP}$-hardness, we will reduce it from the Subset Sum Problem (SSP), which is known to be $\mathcal{NP}$-complete \cite{Cormen2009}. Given an instance of the SSP, consisting of a multiset $S$ of integers and a target sum $b$, we construct an instance of MTMP-DR as follows: set $\alpha=0$, $|\mathcal{A}|=2$, and associate each agent $a_i\in \mathcal{A}$ with $\gamma_i=0$ and $p_i=1/\sqrt{(s_i)}$, where $s_i$ is the element in $S$.
		
		\textbf{The “if and only if” direction:} It is important to note that the construction can be performed in polynomial time. Then, there exists a subset in the SSP instance that sums to the target sum $b$ if and only if the solution utility of the constructed MTMP-DR instance is equal to $1/\sqrt{(s-b)}$, where $s$ is the sum of all elements in $S$.	
	\end{proof}
	
	\begin{figure*}
		\centering
		\includegraphics[width = \textwidth]{Figure/fig2.png}
		\caption{Framework of the Hierarchical Contextual Task Migration Algorithm Based on Multiplex Potential Field.}
		\label{fig:enter-label}
	\end{figure*}
	
	\section{Algorithm Design}
	\label{algorithmdesign}
	The concept of the multiplex potential field is proposed as a guiding principle for performing both inter-network and intra-network task migrations in multiplex networked industrial chains. Fig. \ref{fig:enter-label} shows the framework of our proposed Hierarchical Contextual Task Migration Algorithm Based on Multiplex Potential Field (HCTM-MPF).
	
	\subsection{Hierarchical Task Migration Based on Inter-layer Potential Field}
	
	Building upon the inter-layer potential field, the Inter-Network Task Migration (InterNTM) algorithm employs hierarchical control to select key agents and perform task migration between layers. This approach aims to balance the potential field of layers with disruption-affected agents and other potential fields of unaffected layers. 
%	Fig. \ref{fig3} shows the framework of our proposed InterNTM algorithm.
	
%	\begin{figure}
%		\centering
%		\includegraphics[width = 0.6\textwidth]{Figure/fig3.png}
%		\caption{Framework of the Inter-Network Task Migration Algorithm.}
%		\label{fig3}
%	\end{figure}
	
	\subsubsection{Preliminaries in the InterNTM}
	
	This subsection provides an introduction to the essential concepts and background knowledge related to the InterNTM algorithm.
	
	\begin{definition}
		Key Agents ($A_{key}$): Key agents are identified based on their individual load status and their position within the multiplex network structure, and they are selected to organize task migration between network layers.
	\end{definition}
	
	The fitness of a key agent is directly proportional to the number of network layers it belongs to (denoted as $|\mathcal{A}_i|$) and its importance within the network layers, characterized by its betweenness centrality (denoted as $\mathcal{B}_i$). It is inversely proportional to the load status of the agent (denoted as $1-p_i$). The selection process of key agents can be represented as follows: 
	\begin{equation}
	a_{key}^{[l]}=\arg \max_{a_i \in A^{[l]}}\left(|\mathcal{A}_i|+\mathcal{B}_i\right)/(1-p_i)
	\end{equation}
	
	\begin{definition}
		Inter-layer Potential Field ($\mathcal{PF}_N$): The potential field function $\mathcal{PF}_N (A^{[l]})$ of network layer $A^{[l]}\in \mathcal{A}$ is determined by the weighted combination of the gravitational potential field and repulsive potential field within the network layer. Given the parameters $\delta_N^+$ and $\delta_N^-$ that determine the relative weights of the gravitational potential field and the repulsive potential field within the network layer. The calculation of the inter-layer potential field is performed as follows:
	\end{definition}
	
	\begin{equation}
	\mathcal{PF}_N\left(A^{[l]}\right)=-\delta_N^{+} \cdot \mathcal{PF}_N^{+}\left(A^{[l]}\right)+\delta_N^{-} \cdot \mathcal{PF}_N^{-}\left(A^{[l]}\right)
	\end{equation}
	
	
	\begin{definition}
		Inter-layer Gravitational Potential Field ($\mathcal{PF}_N^+$): The gravitational potential field $\mathcal{PF}_N^+$ of the network layer is inversely related to the average load within the network layer. The monotonically increasing function $\zeta_N(\cdot)$, it can be mathematically represented as follows: 
	\end{definition}
	
	\begin{equation}
	\label{pfn1}
	\mathcal{PF}_N^{+}\left(A^{[l]}\right)=-\zeta_N\left(\sum_{a_i \in A^{[l]}} S_i /\left|A^{[l]}\right|\right)
	\end{equation}
	
	
	\begin{definition}
		Inter-layer Repulsive Potential Field ($\mathcal{PF}_N^-$): The repulsive potential field $\mathcal{PF}_N^-$ of the network layer is positively correlated with the proportion of agents affected by the disruption risk within the network layer relative to all agents. Given the collection of agents $A_e^{[l]}$ within network layer $A^{[l]}$ affected by the disruption risk and a monotonically increasing function $\eta_N(\cdot)$, the repulsive potential field is calculated as follows:
	\end{definition}
	
	\begin{equation}
	\label{pfn2}
	\mathcal{PF}_N^{-}\left(A^{[l]}\right)=
	\begin{cases}
	\eta_N\left(\frac{\left|A_e^{[l]}\right|}{\left|A^{[l]}\right|-\left|A_e^{[l]}\right|}\right), & \left|A^{[l]}\right| \neq\left|A_e^{[l]}\right| \\
	+\infty, & \left|A^{[l]}\right|=\left|A_e^{[l]}\right|
	\end{cases}
	\end{equation}
	
	
	\subsubsection{Detailed Description of the InterNTM} 
	
	In this subsection, a comprehensive and detailed explanation of the InterNTM algorithm is presented. Next, we present the pseudocode for the InterNTM in Algorithm \ref{alg1}.
	
	\begin{algorithm}
		
		\caption{Inter-Network Task Migration}
		\label{alg1}
		\KwIn{Affected agents set $A_e^{\left[l\right]}$ and potential field $\mathcal{PF}_N\left(A^{\left[l\right]}\right)$ in each network layer $A^{\left[l\right]}\in\mathcal{A}$}
		\KwOut{Task allocation set $\mathcal{Q}$ and the total migration cost $c_{inter}$ in inter-network task migration}
		\textbf{Initialize: }$c_{inter}\gets0$\;
		MaxHeap $MH$ = new MaxHeap()\;
		$MH.offer\left(\langle A_e^{\left[l\right]}, \mathcal{PF}_N\left(A^{\left[l\right]}\right)\rangle\right),\forall A^{\left[l\right]}\in\mathcal{A}$\;
		$\langle A_e^\ast,\mathcal{PF}_N^\ast\rangle\gets MH.top()$\;
		PriorityQueue $PQ$ = new PriorityQueue()\;
		$PQ.offer(\langle t_k,s_k\rangle)$\;
		\While{$!MH.isEmpty()\&\frac{\mathcal{PF}_N^\ast}{\bar{\mathcal{PF}_N}}>\epsilon$}{
			$A_{min} \leftarrow \arg \min_{A^{[l]} \in_{\mathcal{A}}}\left\{\mathcal{P} \mathcal{F}_N\left(A^{[l]}\right)\right\}$\;
			\While{$!PQ.isEmpty() \& \frac{\mathcal{PF}_N^*}{\mathcal{PF}_N\left(A_{\min }\right)}>\epsilon$}{
				$<t_k, s_k>\leftarrow PQ.poll()$\;
				$a_{key} \leftarrow$ the key Agent of $A_{min}$\;
				$a_{from} \leftarrow pos\left(t_k\right)$\;
				\textbf{update} $\mathcal{P} \mathcal{F}_N\left(A_{\text {min }}\right), \mathcal{P} \mathcal{F}_N\left(A^{[l]}\right), Q_{\text {from }}, Q_{\text {key }}$\;
				$c_{inter} \gets c_{inter}+c_{from,key}$\;
			}
			\If{$\frac{\mathcal{PF}_N^\ast}{\bar{\mathcal{PF}_N}}\leq \epsilon$}{
				$MH.poll()$\;
				$\langle A_e^\ast,\mathcal{PF}_N^\ast\rangle\gets MH.top()$\;
			}
		}
		
		\textbf{return} $\mathcal{Q},c_{inter}$
	\end{algorithm}
	
	The specific process of InterNTM algorithm can be divided into four main parts:
	
	\begin{itemize}
		\item Initialization Phase (Lines 1-3): In the initialization phase, a max heap is constructed, and the network layers containing agents exposed to the disruption risk are added to the heap. These network layers are sorted in descending order based on the inter-layer potential field.
		\item Task Selection Phase (Lines 4-8): In this phase, the highest priority network layer from the heap is selected. The tasks assigned to the agents affected by the disruption risk within that network layer are identified and added to the set of tasks to be migrated.
		\item Task Migration Phase (Lines 9-14): If the potential field of the network layer at the top of the heap is higher than the average potential field of the system, tasks are migrated to the key agents in the network layer with the lowest potential field. Subsequently, the potential field of the network layer is updated accordingly to reflect the changes in task allocation and the resulting workload distribution.
		\item State Analysis Phase (Lines 15-17): Once the potential field of the network layer at the top of the heap has reached a balanced state, the top element is removed from the heap. The process is then repeated for the next network layer that includes agents affected by the disruption risk, following the same procedure.
	\end{itemize}
	
	
	\subsubsection{Analyses of the InterNTM}
	The time complexity of the InterNTM algorithm is no more than $\mathcal{O}(TK)$ in the case of a multiplex networked industrial chain with $K$ network layers and $T$ tasks in each layer. Next, we prove the rationality of the design of the inter-layer potential field.
	
	\begin{theorem}
		For any two network layers $A^{\left[m\right]}$ and $A^{\left[n\right]}$ in a multiplex networked industrial chain, if the inter-layer potential field $\mathcal{P}\mathcal{F}_N\left(A^{\left[m\right]}\right)>\mathcal{P}\mathcal{F}_N\left(A^{\left[n\right]}\right)$ and the normalized total load ratio $S^{\left[m\right]}/(|A^{\left[m\right]}|-|A_e^{\left[m\right]}|)>S^{\left[n\right]}/(|A^{\left[n\right]}|-|A_e^{\left[n\right]}|)$, where $S^{\left[m\right]}$ is the total load of $A^{\left[m\right]}$ and $S^{\left[n\right]}$ is the total load of $A^{\left[n\right]}$, then migrating tasks from network layer $A^{\left[m\right]}$ to network layer $A^{\left[n\right]}$ will improve the task completion ratio.
	\end{theorem}
	
	\begin{proof}
		Assuming that the weights $\delta_N^+$ and $\delta_N^-$ of the factors affecting the inter-layer potential field are equal, then when $\mathcal{PF}_N(A^{\left[m\right]})>\mathcal{PF}_N(A^{\left[n\right]})$, there are three possible scenarios for the states of $A^{\left[m\right]}$ and $A^{\left[n\right]}$ shown as follows:
		
		\begin{itemize}
			\item $\mathcal{PF}_N^+\left(A^{\left[m\right]}\right)>\mathcal{PF}_N^+\left(A^{\left[n\right]}\right)$, $\mathcal{PF}_N^-\left(A^{\left[m\right]}\right)>\mathcal{PF}_N^-\left(A^{\left[n\right]}\right)$. It can be deduced that the average load of $A^{\left[m\right]}$ is less than that of $A^{\left[n\right]}$, and the proportion of agents affected by the risk of disruption in $A^{\left[m\right]}$ is greater than that in $A^{\left[n\right]}$. Given $S^{\left[m\right]}/(|A^{\left[m\right]}|-|A_e^{\left[m\right]}|)>S^{\left[n\right]}/(|A^{\left[n\right]}|-|A_e^{\left[n\right]}|)$, after inter-layer task migration, the average load of $A^{\left[m\right]}$ will exceed $A^{\left[n\right]}$. Migrating tasks from $A^{\left[m\right]}$ to $A^{\left[n\right]}$ can balance the load between the two network layers and improve the task completion ratio.
			\item $\mathcal{PF}_N^+\left(A^{\left[m\right]}\right)<\mathcal{PF}_N^+\left(A^{\left[n\right]}\right)$, $\mathcal{PF}_N^-\left(A^{\left[m\right]}\right)<\mathcal{PF}_N^-\left(A^{\left[n\right]}\right)$.  The average load of $A^{\left[m\right]}$ is greater than that of $A^{\left[n\right]}$, and the proportion of agents affected by disruption risk in $A^{\left[m\right]}$ is less than that in $A^{\left[n\right]}$. Given $S^{\left[m\right]}/(|A^{\left[m\right]}|-|A_e^{\left[m\right]}|)>S^{\left[n\right]}/(|A^{\left[n\right]}|-|A_e^{\left[n\right]}|)$, after inter-layer task migration, the average load of $A^{\left[m\right]}$ will still exceed $A^{\left[n\right]}$. Migrating tasks from $A^{\left[m\right]}$ to $A^{\left[n\right]}$ can balance the load between the two layers and improve the task completion ratio.
			\item $\mathcal{PF}_N^+\left(A^{\left[m\right]}\right)<\mathcal{PF}_N^+\left(A^{\left[n\right]}\right)$, $\mathcal{PF}_N^-\left(A^{\left[m\right]}\right)>\mathcal{PF}_N^-\left(A^{\left[n\right]}\right)$. It can be deduced that the average load of $A^{\left[m\right]}$ is greater than that of $A^{\left[n\right]}$, and the proportion of agents affected by disruption risk in $A^{\left[m\right]}$ is greater than that in $A^{\left[n\right]}$. In this case, $S^{\left[m\right]}/(|A^{\left[m\right]}|-|A_e^{\left[m\right]}|)>S^{\left[n\right]}/(|A^{\left[n\right]}|-|A_e^{\left[n\right]}|)$ must be satisfied. Migrating tasks from $A^{\left[m\right]}$ to $A^{\left[n\right]}$ can certainly balance the load between the two network layers and improve the task completion ratio.
		\end{itemize}   
		
		In summary, when $\mathcal{PF}_N(A^{\left[m\right]})>\mathcal{PF}_N(A^{\left[n\right]})$ and $S^{\left[m\right]}/(|A^{\left[m\right]}|-|A_e^{\left[m\right]}|)>S^{\left[n\right]}/(|A^{\left[n\right]}|-|A_e^{\left[n\right]}|)$, we can conclude that the migrating tasks from network layer $A^{\left[m\right]}$ to network layer $A^{\left[n\right]}$ can improve the task completion ratio, regardless of which of the above scenarios the potential field states of $A^{\left[m\right]}$ and $A^{\left[n\right]}$ belong to.
	\end{proof}
	
	
	Finally, a detailed analysis of its time complexity is provided in Theorem \ref{TheoremB}.
	
	\begin{theorem}
		For a multiplex networked industrial chain consisting of $K$ network layers, each with $T$ tasks, the time complexity of InterNTM is $\mathcal{O}(TK)$ at most.
		\label{TheoremB}
	\end{theorem}
	\begin{proof}
		The InterNTM algorithm has four phases: initialization phase, task selection phase, task migration phase, and state analysis phase. The detailed time complexity of these phases is as follows:
		
		\begin{itemize}
			\item \textit{Initialization Phase.} The InterNTM algorithm first needs to calculate the potential field for each network layer. For scenarios with $K$ network layers, the complexity of calculating the network layer potential field is $\mathcal{O}(K)$. 
			\item \textit{Task Selection \& Migration Phase.} Tasks need to be migrated to the network layer with the minimum potential field value. The complexity of this algorithm stage is $\mathcal{O}(K)$. The complexity of updating the network layer potential field after migration is $\mathcal{O}(1)$.
			\item \textit{State Analysis Phase.} In the worst case, the number of network layers that require inter-layer task migration does not exceed $K$ layers, and the total number of tasks across all agents in each network layer is $T$. 
		\end{itemize}
		
		In summary, the time complexity of InterNTM in the worst case is $\mathcal{O}(TK)$.
	\end{proof}
	
	
	\subsection{Contextual Task Migration Based on Intra-layer Potential Field}
	\label{InterNTM}
	Based on intra-layer potential field, our proposed Intra-Network Task Migration (IntraNTM) algorithm implements contextual task migration based on the gradient descent direction within the network layer. Fig. \ref{fig4} shows the framework of the IntraNTM algorithm.
	
	\begin{figure*}
		\centering
		\includegraphics[width = \textwidth]{Figure/fig4.png}
		\caption{Framework of the Intra-Network Task Migration Algorithm.}
		\label{fig4}
	\end{figure*}
	
	\subsubsection{Preliminaries in the IntraNTM}
	In this subsection, we provide an overview of the key concepts and background knowledge related to the IntraNTM algorithm. We discuss the basic principles and assumptions underlying the algorithm, as well as the relevant terminology and notation used throughout the paper.
	
	\begin{definition}
		Intra-layer Potential Field ($\mathcal{PF}_{A}$): The potential field function $\mathcal{PF}_A(a_i)$ for each agent $a_i$ is expressed in this paper as a weighted sum of the gravitational potential field determined by the load condition and the repulsive potential field determined by the disruption risk. It can be represented as follows:
	\end{definition}
	\begin{equation}
	\mathcal{P F}_A\left(a_i\right)=-\delta_A^{+} \cdot \mathcal{P} \mathcal{F}_A^{+}\left(a_i\right)+\delta_A^{-} \cdot \mathcal{P} \mathcal{F}_A^{-}\left(a_i\right)
	\end{equation}
	where the parameters $\delta_A^+$ and $\delta_A^-$ represent the weight ratios of the gravitational and repulsive potential fields, respectively.
	
	
	\begin{definition}
		Intra-layer Gravitational Potential Field ($\mathcal{PF}_{A}^+$): The gravitational potential field $\mathcal{PF}_A^+\left(a_i\right)$ exhibits a monotonically increasing nature, which is influenced by the agent's and its contextual agents' reception capabilities to receive tasks. It is represented as follows:
	\end{definition}
	
	\begin{equation}
	\mathcal{PF}_A^{+}\left(a_i\right)=\zeta_A\left(I_i\right)
	\end{equation}
	where $\zeta_A$ denotes the potential gain function, which is a monotonically increasing function. The definition of reception capabilities $I_i$ of agent $a_i$ and the agents in its context is provided below. The gravitational potential field $\mathcal{PF}_A^+$ denotes the agent's degree of attraction that the agent has for tasks. As the task reception capabilities of agent $a_i$ and its context strengthen, their attractiveness for tasks increases as well, making it more likely for other agents to migrate tasks to be executed by $a_i$.
	
	\begin{definition}
		Reception Capabilities of Agent and Its Context ($I_i$): The concept of reception capabilities of agents and their collaborative context in task migration is proposed in this paper, denoted by symbol $I$. The calculation process is as follows:
	\end{definition}
	
	\begin{equation}
	I_i=-\alpha \cdot\left(\frac{S_i}{v_i}+\sum_{a_j \in \Omega^i} \frac{S_j}{v_j}\right)+\beta \cdot\left(p_i+\sum_{a_j \in \Omega^i} p_j\right)
	\end{equation}
	where weight parameters $\alpha$ and $\beta$ align with the weight parameters used in the utility function, while $\Omega^i$ represents the sets of agents within the collaborative context of agent $a_i$. The rule behind the design of \textit{Reception Capabilities of Agent and Its Context} is analyzed in Lemma \ref{lemma1}.
	
	\begin{lemma}
		\label{lemma1}
		For two agents $a_m$ and $a_n$ in a multiplex networked industrial chain, if migrating task $t_k$ to agent $a_m$ yields a higher utility value compared to migrating it to agent $a_n$, then it must be the case that $I_m>I_n$.
	\end{lemma} 
	
	\begin{proof}
		Let $\mathcal{U}(t_k,\ a_m)$ represent the impact of migrating task $t_k$ to agent $a_m$ on sub-objective $\mathcal{U}$ in the utility function, and let $\mathcal{R}(t_k, a_m)$ represent the impact of migrating task $t_k$ to agent $a_m$ on sub-objective $\mathcal{R}$ in the utility function. If both sub-objectives are assigned equal weights, migrating task $t_k$ to agent $a_m$ will yield a higher utility value compared to migrating it to agent $a_n$ only if the following conditions are met: $\mathcal{U}(t_k, a_m)<\mathcal{U}(t_k, a_n)$ and $\mathcal{R}(t_k, a_m)>\mathcal{R}(t_k, a_n)$. Next, we discuss these two conditions separately.
		
		\begin{itemize}
			\item $\mathcal{U}\left(t_k,a_m\right)<\mathcal{U}\left(t_k,a_n\right).$ The factors influencing $\mathcal{U}(\cdot)$ include the task execution cost for the agent. If $\mathcal{U}(t_k,\ a_m)<\mathcal{U}(t_k,\ a_n)$, it implies that the task execution cost for agent $a_m$ and its contextual agents $\Omega^m$ is lower than that for agent $a_n$ and its contextual agents $\Omega^n$, \ie, $S_m/v_m+\sum_{a_j\in\Omega^m}{S_j/v_j}<S_n/v_n+\sum_{a_j\in\Omega^n}{S_j/v_j}$.
			\item $\mathcal{R}(t_k, a_m)>\mathcal{R}(t_k, a_n).$ The factors influencing $\mathcal{R}(\cdot)$ include the task completion ratio for the agent. If $\mathcal{R}(t_k,\ a_m)>\mathcal{R}(t_k,\ a_n)$, it indicates that, for the same received task $t_k$, the task completion ratio for agent $a_m$ and its contextual agents $\Omega^m$ is greater than that for agent $a_n$ and its contextual agents $\Omega^n$, \ie, $p_m+\sum_{a_j\in\Omega^m} p_j>p_n+\sum_{a_j\in\Omega^n} p_j$.
		\end{itemize}
		
		Combining the above two inequalities and using the definition of $I$, it can be concluded that $I_m>I_n$. Therefore, if migrating task $t_k$ to agent $a_m$ yields a higher utility value compared to migrating it to agent $a_n$, it must satisfy the relationship $I_m>I_n$. Thus, Lemma \ref{lemma1} can be proven. 
	\end{proof}
	
	\begin{definition}
		Intra-layer Repulsive Potential Field ($\mathcal{PF}_{A}^-$): The repulsive potential field $\mathcal{PF}_A^-$ reflects the tendency of agents to reject tasks. When there are more agents affected by disruption risk in the context of agent $a_i$, the value of the repulsive potential field increases. The calculation process is represented as follows:
	\end{definition}
	
	\begin{equation}
	\mathcal{P F}_A^{-}\left(a_i\right)=
	\begin{cases}
	\eta_A\left(\frac{\left|\Omega_e^i\right|}{\left|\Omega^i\right|-\left|\Omega_e^i\right|}\right), & \left|\Omega^i\right| \neq\left|\Omega_e^i\right|\\
	+\infty, & \left|\Omega^i\right|=\left|\Omega_e^i\right|
	\end{cases}
	\label{pfa-}
	\end{equation}
	where a large value of $\mathcal{PF}_A^-\left(a_i\right)$ denotes that agent $a_i$ has a decreased potential to acquire task execution capability through task migration and is more likely to reject task reception. It is worth noting that when agent $a_i$ is affected by disruption risk, the value of $\mathcal{PF}_A^-\left(a_i\right)$ becomes infinite.
	
	\begin{definition}
		Gradient of the Intra-layer Potential Field ($\nabla\mathcal{PF}_A^{i\rightarrow j}$): Similar to the definition of the gradient of a potential field in physics \cite{xie2022distributed,huang2022sine,santilli2022multirobot}, the gradient of the intra-layer potential field represents the rate of decrease in the field between agents. The specific definition is as follows:
	\end{definition}
	
	\begin{equation}
	\nabla \mathcal{P F}_A^{i \rightarrow j}=
	\begin{cases}
	\frac{\mathcal{P F}_A\left(a_i\right)-\mathcal{P} \mathcal{F}_A\left(a_j\right)}{c_{i j}}, & \gamma_i=0 \\
	-\frac{\mathcal{P} \mathcal{F}_A\left(a_j\right)}{c_{i j}}, & \gamma_i=1
	\end{cases}
	\end{equation}
	where a higher gradient value $\nabla\mathcal{PF}_A^{i\rightarrow j}$ between agents $a_i$ and $a_j$ indicates that agent $a_j$ has a stronger ability to receive tasks migrated from agent $a_i$.
	
	\subsubsection{Detailed Description of the IntraNTM}
	
	Based on the preliminaries defined above and utilized in the IntraNTM, the pseudo code and detailed algorithmic procedure for the IntraNTM are provided in Algorithm \ref{alg2}.
	
	\begin{algorithm}
		\caption{Intra-Network Task Migration}	
		\label{alg2}
		\KwIn{Affected agents set $A_e^{\left[l\right]}$ and potential field $\mathcal{PF}_A\left(a_i\right)$ of each agent $a_i\in\mathcal{A}$}
		\KwOut{Task allocation set $\mathcal{Q}$ and the total migration cost $c_{intra}$ in intra-network task migration}
		\textbf{Initialize: }${c}_{intra}\gets 0$\;
		\For{each $a_i\in A_e^{\left[l\right]}$}{
			\While{$|Q_i|\neq 0$}{
				Sort tasks in $Q_i$ in ascending order of $s_k$\;
				\For{$t_f \in Q$}{
					$a_j \leftarrow \arg \min\limits_{a_j \in \Omega^i}\left\{\nabla \mathcal{PF}_A^{i \rightarrow j} \mid\gamma_i=1, \gamma_j \neq 1\right\}$\;
					$ Q_j = Q_j \cup \{t_f\}, Q_i=Q_i\setminus\{t_f\}$\;
					\textbf{update} $\mathcal{PF}_A\left(a_j\right),\mathcal{PF}_A\left(\Omega^j\right)$\;
					$c_{intra}\gets c_{intra}+c_{ij}+\operatorname{ITM}(a_j).cost$\;
				}
			}
		}
		\textbf{return} $\mathcal{Q},c_{intra}$\;
		\SetKwProg{Fn}{Function}{}{}
		{\Fn{ \rm \textsc{ITM}{$(a_i)$:} \textcolor{blue}{\Comment{Intra-Context Task Migration}}}{
				\textbf{Initialize: }${cost}\gets 0$\;
				PriorityQueue $PQ$ = new PriorityQueue()\;
				$PQ.offer(\langle a_i, \nabla \mathcal{PF}_A^{i \rightarrow j} \rangle), \forall a_j \in \Omega^i$\;
				$a_m \gets PQ.poll()$\;
				\While{$\nabla\mathcal{PF}_A^{i\rightarrow m}>\vartheta$}{
					$t_{max}\gets \arg\max\limits_{t_k\in Q_i}\{s_k\}$\;
					$Q_m=Q_m\cup\{t_{max}\}, Q_i=Q_i\setminus\{t_{max}\}$\;
					\textbf{update} $\mathcal{PF}_A\left(a_i\right),\ \mathcal{PF}_A\left(a_m\right)$\;
					\textbf{update} $\mathcal{PF}_A\left(\Omega^i\right),\mathcal{PF}_A\left(\Omega^m\right)$\;
					$cost\gets cost+c_{im}+\operatorname{ITM}(a_m).cost$\;
					$PQ.offer(a_m)$\;
					$a_m\gets PQ.poll()$\;
				}
				\textbf{return} $cost$
		}}
	\end{algorithm}
	
	The specific process of the IntraNTM algorithm can be divided into two main parts:
	
	\begin{itemize}
		\item Context-Based Task Migration Phase (Lines 1-8): Each task on the agents affected by disruption risk is sorted in ascending order of their load during the intra-layer task migration process based on the potential field in the context, as indicated in Line 4. The agent with the smallest potential field gradient $\nabla\mathcal{PF}_A^{i\rightarrow j}$ is selected in Line 6, and the task is migrated to that agent in Lines 7 \& 8.
		\item Intra-Context Task Migration Phase (Lines 9\&10): In this phase, task migration is performed within the cooperative context of the agent that has already received tasks. The overall process follows a depth-first diffusion trend along the potential field. First, the potential field gradient values between agent $a_i$ and the set of agents within its context, denoted as $\Omega^i$, are computed. A priority queue is then constructed for the agents within $\Omega^i$ (Lines 12-14). Then, agent $a_m$ who is at the front of the queue, will be selected as the recipient of the migrated task. If the potential field gradient value between $a_m$ and $a_i$ , $\nabla\mathcal{PF}_A^{i\rightarrow m}$, exceeds the threshold $\vartheta$, the algorithm greedily chooses the task with the highest load from agent $a_i$ and migrates it to agent $a_m$. The intra-context task migration algorithm is recursively applied to agent $a_m$ (Lines 15-21). Finally, after the recursive execution, $a_m$ is placed back into the priority queue, and the front agent of the queue is updated (Lines 22\&23). If the potential field gradient value, $\nabla\mathcal{PF}_A^{i\rightarrow m}$, between $a_i$ and the newly selected $a_m$ still satisfies the task migration condition, the process is repeated. Otherwise, the algorithm terminates.
	\end{itemize}
	
	\subsubsection{Analyses of the IntraNTM}
	For the case of a multiplex networked industrial chain with $K$ network layers and $T$ tasks in each layer, the time complexity of the IntraNTM algorithm is no more than $\mathcal{O}(TKN^2)$. Next, we will discuss the rule behind the design of the intra-layer potential field.
	
	\begin{theorem}
		\label{the2}
		In a multiplex networked industrial chain, for any pair of agents $a_m$ and $a_i$, when $\mathcal{PF}_A(a_m)-\mathcal{PF}_A(a_i)>0$ and $I_i\geq I_m$, migrating tasks from agent $a_m$ to agent $a_i$ will yield a higher utility value.
	\end{theorem}
	
	\begin{proof}
		To simplify the proof, the independent variables in Eq. \eqref{pfa-} are denoted as $\rho\left(\cdot\right)$. When $\mathcal{PF}_A(a_m)-\mathcal{PF}_A(a_i)>0$ and $I_i\geq I_m$, the following situations can be observed:
		
		\begin{itemize}
			\item $\rho(a_i)>\rho(a_m)$. The higher reception capabilities of agent $a_i$ and its context, as indicated by $I_i\geq I_m$, suggests that $a_i$ has a lighter load and can execute tasks more efficiently compared to $a_m$. Migrating tasks from $a_m$ to $a_i$ in this situation allows for better load balance among agents and reduces the overall task execution cost. Additionally, when $\rho(a_i)>\rho(a_m)$, the context of $a_i$ has a lower proportion of agents affected by the disruption risk compared to the context of $a_m$. By receiving tasks, $a_i$ benefits from a larger pool of suitable contextual agents, enabling the optimization of load distribution through the recursive Intra-context Task Migration process outlined in Algorithm \ref{alg2}. Thus, migrating tasks from agent $a_m$ to $a_i$ in this scenario yields a higher utility value.
			\item $\rho(a_i) \leq \rho(a_m)$. Similarly, in this scenario, agent $a_i$ has a lighter load and execute tasks more efficiently compared to agent $a_m$. By migrating tasks exclusively from agent $a_m$ to $a_i$, the load among agents can be balanced, resulting in reduced task execution costs. Although $\rho(a_i)\le\rho(a_m)$, indicating a higher proportion of agents affected by disruption risk in the context of $a_i$ compared to $a_m$, migrating tasks solely from $a_m$ to $a_i$ still leads to a higher utility value. Agent $a_i$ does not need to engage in task migration but can still achieve a higher utility value.
		\end{itemize}   
		
		Based on the above analysis, when $\mathcal{P}\mathcal{F}_A(a_m)-\mathcal{P}\mathcal{F}_A(a_i)>0$ and $I_i\geq I_m$, it can be concluded that migrating tasks from agent $a_m$ to $a_i$ will consistently yield a higher utility value, regardless of the specific situations of the agents within their respective contexts. Hence, Theorem \ref{the2} is established.
	\end{proof}
	
	Finally, a detailed analysis of its time complexity is provided in Theorem \ref{AppendixD}. 
	
	\begin{theorem}
		The time complexity of the IntraNTM algorithm in a multiplex networked industrial chain with $K$ network layers, $T$ tasks in each layer, and an average of $N$ agents per network layer is bounded by $\mathcal{O}(TKN^2)$.
		\label{AppendixD}
	\end{theorem}
	
	\begin{proof}
		The IntraNTM algorithm consists of two main parts, the context-based task migration phase and the intra-context task migration phase, with the following complexity analysis:
		
		\begin{itemize}
			\item \textit{Context-based Task Migration Phase.} Firstly, the potential field values of each agent are calculated with a complexity of $\mathcal{O}(N)$ for $N$ agents within the network layer. Then, task migration occurs along the direction of decreasing potential field values for agents impacted by disruption risk. This migration strategy utilizes a depth-first search approach with a complexity of $\mathcal{O}(N)$ for finding task migration paths. The number of tasks to be migrated within the network layer is limited to $T$.
			\item \textit{Intra-context Task Migration Phase.} In the worst-case scenario, each agent undergoes recursive context-based task migration in the direction of decreasing potential field values. Consequently, the total algorithm complexity for task migration within $K$ network layers in the worst-case scenario is $\mathcal{O}(TKN^2)$.
		\end{itemize}
		
		In summary, the time complexity of IntraNTM is bounded by $\mathcal{O}(TKN^2)$. 
	\end{proof}
	
	\section{Experiments}
	\label{experiments}
	
	\subsection{Experimental Settings}
	The experiments were conducted on a PC with an AMD Ryzen 7 3800X processor, Windows 10 operating system, and 16GB RAM. The experimental code was implemented using Java 11. A total of 5000 instances were run, and the results are reported as averages. The statistical significance of each record is at the 95\% confidence level, and the error bars in the experimental results represent the 95\% confidence interval.
	
	\subsection{Benchmark Algorithms}
	The selected benchmark algorithms in this paper are as follows:
	
	\begin{itemize}
		\item Hierarchical Contextual Task Migration Based on Multiplex Potential Field (HCTM-MPF): This algorithm is proposed in Section \ref{algorithmdesign} of this paper. The threshold parameters $\vartheta$ and $\epsilon$ in HCTM-MPF are set to 0.3 and 1.35, respectively, estimated from the distribution of simulation data and experimental results. The ratio parameters $\delta_A^+$ and $\delta_N^+$ for the gravitational potential fields are equal to the weight parameter $\alpha$, while $\delta_A^-$ and $\delta_N^-$ are equal to the weight parameter $\beta$. This allows the HCTM-MPF algorithm to adapt to different values of $\alpha$ and $\beta$ in the optimization sub-objectives.
		\item Context-Aware Task Migration (CATM): This is a conventional approach used in previous multiagent task migration problems within simplex networks \cite{chowdhury2018context}. In CATM, the agent requesting migration sends a task migration request to agents within its context. After receiving feedback, the agent migrates the task to other agents that meet the load threshold requirement and possess the highest utility for task completion.
		\item Key-Based Task Migration (KBTM): This is an effective method for solving task redistribution (which can be generalized as task migration) in multiplex networks \cite{jiang2022batch}. It utilizes a hybrid control model of a leader agent and a distributed network. The leader agent is responsible for managing group information and collaboration among groups, while collaborative task allocation within groups is still performed in a distributed manner.
		\item Optimal Result (OPT): This term refers to the optimal solution obtained by solving the integer programming model presented in Section \ref{iip} using the CPLEX solver.
	\end{itemize}
	
	\begin{figure*}[htb]
		\centering
		\subfigure[]{
			\begin{minipage}[t]{0.31\linewidth} 
				\centering
				\includegraphics[width=1.7in]{Figure/fig5a.png}
				\label{casefig5a}
			\end{minipage}
		}
		\subfigure[]{
			\begin{minipage}[t]{0.31\linewidth}
				\centering
				\includegraphics[width=1.7in]{Figure/fig5b.png}
				\label{casefig5b}
			\end{minipage}
		}
		\subfigure[]{
			\begin{minipage}[t]{0.31\linewidth}
				\centering
				\includegraphics[width=1.7in]{Figure/fig5c.png}
				\label{casefig5c}
			\end{minipage}
		}
		
		
		\subfigure[]{
			\begin{minipage}[t]{0.31\linewidth}
				\centering
				\includegraphics[width=1.7in]{Figure/fig5d.png}
				\label{casefig5d}
			\end{minipage}
		}
		\subfigure[]{
			\begin{minipage}[t]{0.31\linewidth} 
				\centering
				\includegraphics[width=1.7in]{Figure/fig5e.png}
				\label{casefig5e}
			\end{minipage}
		}
		
		\subfigure[]{
			\begin{minipage}[t]{0.31\linewidth}
				\centering
				\includegraphics[width=1.7in]{Figure/fig5f.png}
				\label{casefig5f}
			\end{minipage}
		}
		\subfigure[]{
			\begin{minipage}[t]{0.31\linewidth}
				\centering
				\includegraphics[width=1.7in]{Figure/fig5g.png}
				\label{casefig5g}
			\end{minipage}
		}
		\subfigure[]{
			\begin{minipage}[t]{0.31\linewidth}
				\centering
				\includegraphics[width=1.7in]{Figure/fig5h.png}
				\label{casefig5h}
			\end{minipage}
		}
		
		\centering
		\caption{Test on the objectives. Test on the task completion ratio: (a)-(d); test on the task execution cost increase ratio: (e)-(h).}
		\label{fig5}
	\end{figure*}
	
	
	
	\subsection{Test on Task Completion Ratio}
	In this subsection, we evaluate the TCRs achieved by different benchmark algorithms under varying conditions, including the number of tasks, the number of agents, the degree of disruption risk, and the number of network layers. 
	
	\subsubsection{Case Study}
	A case study was conducted using weight parameters $\alpha=0.3$ and $\beta=0.7$. The multiplex network consisted of 4 layers with a randomly generated connected graph structure. There were 16 agents distributed randomly across the network layers, with 20\% of agents affected by disruption risk. The number of tasks was set to 24, following a normal distribution in terms of task size. The TCRs were evaluated by varying different parameters to analyze the performance of the benchmark algorithms.
	
	\subsubsection{Test on Influence of Number of Tasks}
	In this series of tests, the number of tasks varied from 6 to 30, with the task sizes following a normal distribution. As shown in Fig. \ref{casefig5a}, as the number of tasks increased, the TCRs for different algorithms showed a decreasing trend. The OPT and HCTM-MPF algorithms outperformed the KBTM algorithm and significantly outperformed the CATM algorithm. The  results achieved by HCTM-MPF algorithm closest to the optimal solution.
	
	With an increasing number of tasks, the load on each agent increased. This led to an increase in the number of tasks that needed to be migrated for agents affected by disruption risk, as well as the tasks that needed to be executed on normal agents. Consequently, there was a higher likelihood of overloading normal agents after executing the task migration strategy, resulting in a continuous decline in TCR.
	
	\subsubsection{Test on Influence of Number of Agents}
	
	In this series of tests, the number of agents varied from 8 to 24. As shown in Fig. \ref{casefig5b}, the TCRs of different algorithms shown an upward trend with an increasing number of agents. However, this upward trend started to plateau when the number of agents reached a certain threshold. The proposed HCTM-MPF algorithm consistently outperformed the KBTM algorithm and significantly outperformed the CATM algorithm, approaching the optimal solution. 
	
	With an increasing number of agents, the load per agent decreased. Although the number of agents affected by disruption risk also increased, the overall number of tasks requiring migration remained constant. As a result, more agents were available to receive tasks, and the likelihood of agent overload decreased, leading to an increase in TCR. However, when the number of agents reached a certain threshold, the sub-objective of minimizing TECIR constrained the ability to fully utilize all agents for load balancing, causing the upward trend to plateau. 
	
	In conclusion, this study found that increasing the number of agents had a positive impact on TCR within a certain range. However, once the system reached a certain number of agents, further increasing the agent count yielded limited improvement in TCR.
	
	\subsubsection{Test on Influence of Degree of Disruption Risk}
	
	In this series of tests, the proportion of agents affected by disruption risk ranged from 0.1 to 0.3. As depicted in Fig. \ref{casefig5c}, the TCRs of different algorithms exhibited a decreasing trend as the proportion of agents affected by disruption risk increased. The proposed HCTM-MPF algorithm consistently outperformed the KBTM and CATM algorithms, with its advantage becoming more pronounced as the proportion of agents affected by disruption risk increased.
	
	With a higher proportion of agents affected by disruption risk and a fixed number of tasks, the number of tasks requiring migration increased. This led to a higher likelihood of overload for receiving agents and a greater number of tasks being abandoned to avoid agent overload, resulting in a decline in TCR. The HCTM-MPF algorithm effectively leveraged the collective intelligence of the global agents to distribute tasks executed in the multiplex networked industrial chains.
	
	\subsubsection{Test on Influence of Number of Network Layers}
	
	In this series of tests, the number of network layers ranged from 2 to 6. As shown in Fig. \ref{casefig5d}, the TCRs of different algorithms decreased as the number of network layers increased. The HCTM-MPF algorithm outperformed the two comparison algorithms in terms of TCR, with a more gradual decline than the CATM algorithm.
	
	With more network layers, a higher proportion of agents affected by disruption risk could potentially impact multiple layers, increasing the likelihood of network layer overload affecting agents. This led to a downward trend in TCR. However, the HCTM-MPF algorithm effectively addressed this issue through hierarchical contextual task migration, resulting in a slower decline in TCR.
	
	\begin{figure*}[htb]
		\centering
		\subfigure[$\alpha=0.1,\beta=0.9$]{
			\begin{minipage}[t]{0.3\linewidth} 
				\centering
				\includegraphics[width=1.7in]{Figure/fig6a.png}
				\label{casefig6a}
			\end{minipage}
		}
		\subfigure[$\alpha=0.3,\beta=0.7$]{
			\begin{minipage}[t]{0.3\linewidth}
				\centering
				\includegraphics[width=1.7in]{Figure/fig6b.png}
				\label{casefig6b}
			\end{minipage}
		}
		\subfigure[$\alpha=0.5,\beta=0.5$]{
			\begin{minipage}[t]{0.3\linewidth}
				\centering
				\includegraphics[width=1.7in]{Figure/fig6c.png}
				\label{casefig6c}
			\end{minipage}
		}
		
		\subfigure[$\alpha=0.1,\beta=0.9$]{
			\begin{minipage}[t]{0.3\linewidth} 
				\centering
				\includegraphics[width=1.7in]{Figure/fig6d.png}
				\label{casefig6d}
			\end{minipage}
		}
		\subfigure[$\alpha=0.3,\beta=0.7$]{
			\begin{minipage}[t]{0.3\linewidth}
				\centering
				\includegraphics[width=1.7in]{Figure/fig6e.png}
				\label{casefig6e}
			\end{minipage}
		}
		\subfigure[$\alpha=0.5,\beta=0.5$]{
			\begin{minipage}[t]{0.3\linewidth}
				\centering
				\includegraphics[width=1.7in]{Figure/fig6f.png}
				\label{casefig6f}
			\end{minipage}
		}
		\centering
		\caption{Test on the influence of weight parameters on the objectives. Test on the influence on the task completion ratios: (a)-(c); test on the influence on the task execution cost increase ratio: (d)-(f).}
	\end{figure*}
	
	
	% \begin{figure}
	%     \centering
	%     \begin{minipage}[t]{\linewidth}
	%         \centering
	%         \includegraphics[width=3in]{Figure/fig7.png}
	%         \caption{Test on the running time.}
	%         \label{casefig7}
	%     \end{minipage}
	%     \begin{minipage}[t]{\linewidth}
	%         \centering
	%         \includegraphics[width=3in]{Figure/fig8.png}
	%         \caption{Test on the utility.}
	%         \label{casefig8}
	%     \end{minipage}
	% \end{figure}
	
	\subsection{Test on Task Execution Cost Increase Ratio}
	
	In this subsection, we examine how different benchmark algorithms perform in terms of the TECIR under various conditions, including variations in the number of tasks, number of agents, degree of disruption risk, and number of network layers.
	
	\subsubsection{Case Study}
	
	A case study was conducted using weight parameters $\alpha=0.3$ and $\beta=0.7$. The multiplex network consisted of 4 layers with a randomly generated connected graph structure. There were 16 agents distributed randomly across the network layers, with 20\% of agents affected by disruption risk. The number of tasks was set to 24, following a normal distribution in terms of task size. The TECIRs were evaluated by varying different parameters to analyze the performance of the benchmark algorithms.
	
	\subsubsection{Test on Influence of Number of Tasks}
	
	In this series of tests, the number of tasks ranged from 6 to 30, with task sizes following a normal distribution. As shown in Fig. \ref{casefig5e}, as the number of tasks increased, the TECIRs for different algorithms exhibited an upward trend. The HCTM-MPF algorithm had a lower TECIR than the two benchmark algorithms but a higher TECIR than the OPT algorithm.
	
	With an increasing number of tasks in the system, while keeping the proportion of agents affected by disruption risk constant, the number of tasks requiring migration increased. The HCTM-MPF algorithm and the OPT algorithm dynamically abandoned some tasks to reduce the task execution cost and alleviate the impact of agent overloading caused by disruption risk. Consequently, the TECIRs of the HCTM-MPF algorithm and the OPT algorithm were lower than those of the two comparison algorithms.
	
	\subsubsection{Test on Influence of Number of Agents}
	
	In this series of tests, the number of agents varied from 8 to 24. As depicted in Fig. \ref{casefig5f}, the TECIR showed an overall decreasing trend  as the number of agents increased, but it reached a plateau after a certain threshold.
	
	With more agents, while maintaining a constant proportion of agents affected by disruption risk, the number of tasks requiring migration remained unchanged. However, the number of agents capable of receiving tasks increased, reducing the likelihood of agent overload. Consequently, more tasks were migrated to agents with lower execution costs, resulting in decreased task completion costs. However, beyond a certain agent count, the system cannot fully exploit the potential of all agents due to the task migration cost limitations. As a result, the task execution costs ceased to decrease.
	
	\subsubsection{Test on Influence of Degree of Disruption Risk}
	
	In this series of tests, the proportion of agents affected by disruption risk ranged from 0.1 to 0.3. As shown in Fig. \ref{casefig5g}, as this proportion increased, the TECIR increased for all algorithms. The proposed HCTM-MPF algorithm consistently outperformed the KBTM and CAM algorithms in terms of TECIR.
	
	As the proportion of agents affected by disruption risk increased, more tasks needed to be migrated, while the number of agents capable of receiving tasks decreased. This led to an upward trend in TECIR. The HCTM-MPF algorithm intelligently chose not to execute certain tasks with high completion costs, resulting in a lower TECIR.
	
	\subsubsection{Test on Influence of Number of Network Layers}
	
	Test on Influence of Number of Network Layers: In this series of tests, the number of network layers ranged from 2 to 6. As shown in Fig. \ref{casefig5h}, as the number of layers increased, the TECIR of the CATM algorithm exhibited irregular fluctuations, while the TECIRs of the HCTM-MPF and KBTM algorithms showed an upward trend.
	
	With more network layers, the HCTM-MPF and KBTM algorithms performed additional task migrations between layers to address the network layer overload. This resulted in an increase in TECIR. In contrast, the CATM algorithm, which does not consider network layer attributes, showed irregular changes in TECIR.
	
	\subsection{Test on Weight Parameters}
	
	In this subsection, we assess the performance of different benchmark algorithms in terms of TCR and TECIR while considering variations in the weight parameters of the objective functions.
	
	\subsubsection{Case Study}
	
	A case study was conducted using a multiplex network comprising 4 layers with a randomly generated connected graph structure. The system consisted of 16 agents randomly distributed across the network layers, with 20\% of the agents affected by disruption risk. The number of tasks was set to 24, following a normal distribution in terms of task size. The performance of the benchmark algorithms was evaluated by analyzing TCR and TECIR under various weight parameter settings.
	
	\subsubsection{Test on Influence of Task Completion Ratio} As depicted in Fig. \ref{casefig6a}-\ref{casefig6c}, variations in the weight parameters $\alpha\in (0,1]$ and $\beta\in (0,1]$ lead to different patterns in the TCR. For a given number of tasks, lower $\beta$ correspond to lower TCRs for the benchmark algorithms analyzed. However, irrespective of these parameter changes, the proposed HCTM-MPF algorithm consistently outperforms the KBTM and CATM algorithms, closely approaching the experimental results obtained with the OPT algorithm.
	
	Different weight parameter settings influence the relative significance of TCR within the optimization objectives of adaptive task migration strategies. The selected benchmark algorithms produce migration strategies that prioritize TCR optimization to varying degrees. The experimental findings highlight the adaptability of the proposed HCTM-MPF algorithm in adjusting task migration strategies to enhance TCR as needed.
	
	\subsubsection{Test on Influence of Task Execution Cost Increase Ratio}
	
	As depicted in Fig. \ref{casefig6d}-\ref{casefig6f}, the selected four benchmark algorithms yielded varying TECIRs when considering the same number of tasks. Notably, a higher value of the weight parameter $\alpha$, associated with the sub-objective of TECIR, led to lower TECIRs across all algorithms. Regardless of parameter variations, the proposed HCTM-MPF algorithm consistently exhibited lower TECIRs compared to the KBTM and CATM algorithms, approaching the performance of the OPT algorithm.
	
	A higher value of $\alpha$ indicated a stronger preference for migration strategies aimed at reducing task execution costs within the adaptive task migration process. Consequently, the selected benchmark algorithms achieved lower TECIRs. These findings emphasize the adaptability of the proposed HCTM-MPF algorithm in adjusting task migration strategies based on optimization objectives, effectively minimizing the additional task execution costs associated with disruption risks.
	
	\begin{figure}[htbp]
		\centering
		\begin{minipage}{0.48\textwidth}
			\centering
			\includegraphics[width=2.5in]{Figure/fig7.png}
			\caption{Test on the running time.}
			\label{casefig7}
		\end{minipage}\hfill
		\begin{minipage}{0.48\textwidth}
			\centering
			\includegraphics[width=2.5in]{Figure/fig8.png}
			\caption{Test on the utility.}
			\label{casefig8}
		\end{minipage}
	\end{figure}

%	\begin{figure}
%		\centering
%		\includegraphics[width=2.5in]{Figure/fig7.png}
%		\caption{Test on the running time.}
%		\label{casefig7}
%	\end{figure}
%	\begin{figure}
%		\centering
%		\includegraphics[width=2.5in]{Figure/fig8.png}
%		\caption{Test on the utility.}
%		\label{casefig8}
%	\end{figure}
	
	\subsection{Test on Running Time}
	
	In this subsection, we evaluate the computational efficiency of different benchmark algorithms by analyzing their running time under varying problem scales, specifically the number of tasks.
	
	\subsubsection{Case Study} A case study was conducted by utilizing weight parameters $\alpha=0.3$ and $\beta=0.7$. The multiplex network consisted of 4 layers with a randomly generated connected graph structure. The system consisted of 16 agents randomly distributed across the network layers, with 20\% of agents affected by disruption risk. The running time was assessed by varying different problem scales to evaluate the efficiency of the benchmark algorithms.
	
	
	\subsubsection{Test on Influence of Different Problem Scales} 
	
	In Fig. \ref{casefig7}, it can be observed that as the number of tasks increases, the running time of the OPT algorithm grows substantially and surpasses the other benchmark algorithms. Meanwhile, the running time of the HCTM-MPF algorithm and the other benchmark algorithms remains relatively consistent.
	
	The exponential growth of the search space needed by the OPT algorithm when dealing with a larger number of tasks leads to a significant increase in computation time. On the other hand, the HCTM-MPF algorithm and other benchmark algorithms utilize heuristic approaches based on greedy principles, enabling them to achieve efficient computation and shorter running times.
	
	\subsection{Test on Utility}
	
	In this subsection, we evaluate the effectiveness of different benchmark algorithms by analyzing their utility under various parameters. Although we have already discussed the sub-objectives within the utility function in detail earlier, our focus in this subsection is on assessing the impact of problem scales, particularly the number of tasks, on the utility values.
	
	\subsubsection{Case Study}
	
	A case study was conducted with varying problem scales to evaluate the efficiency of the benchmark algorithms. The system included 16 agents distributed across 4 layers of a multiplex network, with 20\% of agents affected by disruption risk. The utility was assessed using weight parameters $\alpha=0.3$ and $\beta=0.7$.
	
	\subsubsection{Test on Influence of Different Problem Scales}
	
	As depicted in Fig. \ref{casefig8}, the utility values of the different algorithms, which represent the optimisation objectives, show a decreasing trend as the number of tasks increases. Notably, the proposed HCTM-MPF algorithm consistently achieves utility values that closely approximate those of the OPT algorithm, outperforming the performance of the two benchmark algorithms.
	
	With an increasing number of tasks, it has been observed from previous experiments that TECIR rises while TCR declines. As a result, the combined utility function values decrease accordingly. However, irrespective of the task quantity, the HCTM-MPF algorithm leverages a potential field framework by considering both the network layers and agents. It employs a context-aware task migration strategy based on depth-first diffusion, accounting for the impact of disruption risk across multiple network layers. This approach yields superior utility function values compared to the two comparison algorithms and approaches the performance of the OPT algorithm.
	
	
	\section{Conclusions and Future Work}
	\label{conclusions}
	This paper explores the challenge of adaptive task migration in multiplex networked industrial chains under the risk of chain disruption. These chains are characterized by the simultaneous impact of disruption risk on multiple network layers and product agents, resulting in a discrepancy between agent capabilities and network layer load status.
	
	To address this issue, a multiplex potential field model is proposed to describe the impact of disruption risk on product agents and their corresponding network layers. Based on this model, a hierarchical contextual task migration algorithm based on the multiplex potential field is developed for both inter-layer and intra-layer task migration, considering the agents exposed to the disruption risk. The proposed algorithm aims to minimize the task execution cost while maximizing the task completion ratio under the influence of disruption risks. It effectively mitigates the losses caused by incomplete tasks due to the chain disruption risk.
	
	Extensive comparative experiments are conducted, considering the metrics of the task execution cost increase ratio, task completion ratio, algorithm running time, and utility function of optimization objectives. The results demonstrate that the proposed task migration algorithm is better suited for multiplex networked industrial chain scenarios with chain disruption risk, showing superior optimization performance. Moreover, the algorithm exhibits adaptability to different experimental settings by dynamically adjusting the task migration strategy. Notably, in all experimental results, it is observed that the proposed algorithm demonstrates more significant advantages compared to the comparative algorithms, particularly when the number of product agents affected by the disruption risk increases.
	
	In future research, the problem of adaptive task migration in multiplex networked industrial chains with heterogeneous product agents will be explored. The product agents considered in this paper are assumed to have only one type of resource, with homogeneous resources across all agents, differing only in their ability to perform tasks. However, it is possible to extend the scope of this study to scenarios where product agents have heterogeneous resources, with different types of resources available to different agents. In such scenarios, the interaction among product agents in multiplex networked industrial chains can involve not only task migration but also resource collaboration, enabling more efficient task completion.
	
	
	
	\section{Acknowledgments}
	This work was supported by the National Key Research and Development Program of China (No.2022YFB3304400), the National Natural Science Foundation of China (Nos. 62476121, 62303111, 62076060, and 61932007), Guangxi Science and Technology Major Program (No. AA24206003), the Key Research and Development Program of Guangxi (AB2410317) and the Key Research and Development Program of Jiangsu Province of China (No. BE2022157).
	
	
	
	\bibliographystyle{unsrt}
	% \bibliographystyle{ACM-Reference-Format}
	\bibliography{ACMexample}%%我们的例子应该是\bibliography{cited}
	
\end{document}


